\documentclass[12pt]{article}
\usepackage[utf8]{inputenc}
\usepackage{upquote}
\usepackage[margin=20mm]{geometry} 
\usepackage{amsmath,amsthm,amssymb}
\usepackage{graphicx}
\usepackage{listings}
\newenvironment{statement}[2][Statement]{\begin{trivlist}
\item[\hskip \labelsep {\bfseries #1}\hskip \labelsep {\bfseries #2.}]}{\end{trivlist}}
\usepackage{xcolor}

\usepackage{subfigure}


% Listings package for code rendering (No external dependencies)
\usepackage{listings}  
\usepackage{xcolor}   % Color support
\usepackage{tcolorbox} % Box for better appearance

% Define custom colors for code highlighting
\definecolor{codegreen}{rgb}{0,0.6,0}
\definecolor{codegray}{rgb}{0.5,0.5,0.5}
\definecolor{codepurple}{rgb}{0.58,0,0.82}
\definecolor{backcolour}{rgb}{0.95,0.95,0.92}


\lstset{frame=tb,
    language=Python,
    backgroundcolor=\color{backcolour},   
    commentstyle=\color{codegreen},
    keywordstyle=\color{magenta},
    numberstyle=\tiny\color{codegray},
    stringstyle=\color{codepurple},
    basicstyle=\ttfamily\footnotesize,
    breakatwhitespace=false,         
    breaklines=true,                 
    keepspaces=true,                 
    numbers=left,       
    numbersep=5pt,                  
    showspaces=false,                
    showstringspaces=false,
    showtabs=false,                  
    tabsize=2,
}





\title{Assignment 7}


\author{Author \\
 Wanjing Hu / fng685@alumni.ku.dk  \\
 Shuangcheng Jia / bkg713@alumni.ku.dk/   \\
 Zhigao Yan / sxd343@alumni.ku.dk  \\
} 

\begin{document}
\maketitle

\section{Segmentation}
%shuangcheng

\section{Image Transform}
%shuangcheng

\section{The Hough Transform}
%zhigao
\subsection{}
\begin{lstlisting}
    def hough_transform(edge_image, theta_step=1, rho_step=1, threshold=100):
    # Chose range and discretization of ρ and θ.
    rows, cols = edge_image.shape
    diagonal = int(np.ceil(np.sqrt(rows**2 + cols**2)))  # Edge image diagonal length
    thetas = np.deg2rad(np.arange(0, 180, theta_step))   # theta : 0-180°
    rhos = np.arange(-diagonal, diagonal, rho_step)      # rho : [-d, d]
    
    # Create accumulator 2D array (Hough Space)
    accumulator = np.zeros((len(rhos), len(thetas)), dtype=np.uint64)
    
    # Iterate over all edge pixels
    edge_points = np.argwhere(edge_image)  # Get all the edge coordinates [y, x]
    
    for (y, x) in edge_points:            # Iterate over each edge point
        for theta_idx, theta in enumerate(thetas):  
            # Calculate the corresponding value of ρ
            rho = x * np.cos(theta) + y * np.sin(theta)
            
            # Ensure that the accumulator array index cannot be negative
            rho_idx = int(round((rho + diagonal) / rho_step))
            
            # Updating the accumulator (to ensure that it does not go out of bounds)
            if 0 <= rho_idx < len(rhos):
                accumulator[rho_idx, theta_idx] += 1
    
    # Lines correspond to accumulator values larger than a certain threshold.
    lines = []
    for rho_idx in range(accumulator.shape[0]):
        for theta_idx in range(accumulator.shape[1]):
            if accumulator[rho_idx, theta_idx] >= threshold:
                lines.append( (rhos[rho_idx], thetas[theta_idx]) )
    
    return lines, accumulator
\end{lstlisting}
The code is written strictly following the steps on slide.
Firstly input the edge image, the step size and the threshold value. Determine the range of $ \theta $ and $ \rho $ according to the steps in slide.
Then the first iteration is performed and the corresponding $ \rho $ is computed. the second iteration gets the lines that are greater than the threshold.


Computational complexity (two iterations): 
\[ O(E \times T) + O(R \times T) \]
E: indicates the number of edge pixels.
T: The number of $ \theta $
R: The number of all calculated $ \rho $ in the accumulator

\subsection{}
\begin{figure}[ht]
    \centering
    \includegraphics[width=0.7\textwidth]{pics/A7_3.2_1.png} 
    \caption{Hough Transform by my implementation}
    \label{fig: Figure 1}
\end{figure}
\begin{figure}[ht]
    \centering
    \includegraphics[width=0.7\textwidth]{pics/A7_3.2_2.png} 
    \caption{Hough Transform by skimage}
    \label{fig: Figure 2}
\end{figure}
First of all, my implementation and skimga's implementation output the same Hough transform.
But as can be seen in Figures \ref{fig: Figure 1} and \ref{fig: Figure 2}, my method outputs more lines for the same threshold.
I think there may be a difference in the method of calculation, skimage's method may filter adjacent peaks resulting in fewer lines.
\subsection{}
\begin{figure}[ht]
    \centering
    \includegraphics[width=0.8\textwidth]{pics/A7_3.3.png} 
    \caption{Hough circle by skimage}
    \label{fig: Figure 3}
\end{figure}
First use the Canny method to get the edge graph.
Then set a radius range, based on the size of the coin in the picture. 
The next step is to call the hough-circle method to get the accumulator, and finally call hough-circle-peaks to get the most obvious circle and draw it.
\subsection{}
\section{Morphology}
%wanjing

\end{document}
