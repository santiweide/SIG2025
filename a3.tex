\documentclass[12pt]{article}
\usepackage[utf8]{inputenc}
\usepackage{upquote}
\usepackage[margin=20mm]{geometry} 
\usepackage{amsmath,amsthm,amssymb}
\usepackage{graphicx}
\usepackage{listings}
\newenvironment{statement}[2][Statement]{\begin{trivlist}
\item[\hskip \labelsep {\bfseries #1}\hskip \labelsep {\bfseries #2.}]}{\end{trivlist}}
\usepackage{xcolor}

\usepackage{subfigure}


% Listings package for code rendering (No external dependencies)
\usepackage{listings}  
\usepackage{xcolor}   % Color support
\usepackage{tcolorbox} % Box for better appearance

% Define custom colors for code highlighting
\definecolor{codegreen}{rgb}{0,0.6,0}
\definecolor{codegray}{rgb}{0.5,0.5,0.5}
\definecolor{codepurple}{rgb}{0.58,0,0.82}
\definecolor{backcolour}{rgb}{0.95,0.95,0.92}


\lstset{frame=tb,
    language=Python,
    backgroundcolor=\color{backcolour},   
    commentstyle=\color{codegreen},
    keywordstyle=\color{magenta},
    numberstyle=\tiny\color{codegray},
    stringstyle=\color{codepurple},
    basicstyle=\ttfamily\footnotesize,
    breakatwhitespace=false,         
    breaklines=true,                 
    keepspaces=true,                 
    numbers=left,       
    numbersep=5pt,                  
    showspaces=false,                
    showstringspaces=false,
    showtabs=false,                  
    tabsize=2,
}


\title{Assignment 3}

\begin{document}
\maketitle


\section{1 Complex numbers}
%wanjing

\subsection{} % 1.1

\textbf{(i)} 

\begin{equation}
\begin{aligned}
d &=(a_1 + i a_2) + (b_1 + i b_2) = (a_1 + b_1) + i\,(a_2 + b_2).
\end{aligned}
\end{equation}

\textbf{(ii)}
\begin{equation}
\begin{aligned}
d &=(a_1 + i a_2) - (b_1 + i b_2) = (a_1 - b_1) + i\,(a_2 - b_2).
\end{aligned}
\end{equation}

\textbf{(iii)} 
\begin{equation}
\begin{aligned}
d &=(a_1 + i a_2)(b_1 + i b_2) \\
&= a_1b_1 + i a_1b_2 + i a_2b_1 + i^2 a_2b_2\\
&= (a_1b_1 - a_2b_2) + i\,(a_1b_2 + a_2b_1).
\end{aligned}
\end{equation}

\textbf{(iv)} For the quotient 
\begin{equation}
\begin{aligned}
d &= \frac{a}{b} \\
&= \frac{a_1 + i\,a_2}{b_1 + i\,b_2}\\
&= \frac{(a_1 + i\,a_2)(b_1 - i\,b_2)}{(b_1 + i\,b_2)(b_1 - i\,b_2)}\\
&= \frac{(a_1 + i\,a_2)(b_1 - i\,b_2)}{b_1^2 + b_2^2} \\
&= \frac{a_1b_1 + a_2b_2 + i\,(a_2b_1 - a_1b_2)}{b_1^2 + b_2^2}\\
&= \frac{a_1b_1 + a_2b_2}{b_1^2+b_2^2} + i\,\frac{a_2b_1 - a_1b_2}{b_1^2+b_2^2}.
\end{aligned}
\end{equation}


\subsection{} % 1.2
by $\sqrt{-3} = \sqrt{3}\,i$ , we have $d = 0 + i\,\sqrt{3}$. 

\subsection{} % 1.3
\textbf{(i)} 
\begin{equation}
\begin{aligned}
d &= a \cdot b = a_r e^{i a_\theta} \cdot b_r e^{i b_\theta} = a_r b_r e^{i(a_\theta + b_\theta)}
\end{aligned}
\end{equation}


\textbf{(ii)} 

\begin{equation}
\begin{aligned}
d &=  \frac{a}{b} = \frac{a_r e^{i a_\theta}}{ b_r e^{i b_\theta}} =  \frac{a_r}{b_r} e^{i(a_\theta - b_\theta)}
\end{aligned}
\end{equation}

\subsection{} %1.4
\begin{equation}
\begin{aligned}
\overline{a_r e^{i a_\theta}} = a_r e^{-i a_\theta}.
\end{aligned}
\end{equation}

\subsection{}

Let 
\[
a = a_r e^{i a_\theta} = a_r (\cos a_\theta + i\sin a_\theta)
\]
and
\[
b = b_r e^{i b_\theta} = b_r (\cos b_\theta + i\sin b_\theta).
\]

\textbf{(i)}
\begin{equation}
\begin{aligned}
d &= a + b\\
& = \Bigl(a_r\cos a_\theta + b_r\cos b_\theta\Bigr) + i\Bigl(a_r\sin a_\theta + b_r\sin b_\theta\Bigr).
\end{aligned}
\end{equation}


\textbf{(ii)} For 
\begin{equation}
\begin{aligned}
d &= a + b\\
& = \Bigl(a_r\cos a_\theta - b_r\cos b_\theta\Bigr) + i\Bigl(a_r\sin a_\theta - b_r\sin b_\theta\Bigr).
\end{aligned}
\end{equation}
\subsection{}

The polar form is given by
\[
a = r\, e^{i\theta},
\]
where
\[
r = \sqrt{a_1^2 + a_2^2} \quad \text{and} \quad \theta = \arctan\!\left(\frac{a_2}{a_1}\right)
\]
So
\[
a = \sqrt{a_1^2+a_2^2}\; e^{i\arctan\!\left(\frac{a_2}{a_1}\right)}.
\]


\section{2 Fourier Transform – Theory}
%zhigao

\section{3 Fourier Transform – In Practice}
%shuangcheng



\end{document}
