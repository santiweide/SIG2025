\documentclass[12pt]{article}
\usepackage[utf8]{inputenc}
\usepackage{upquote}
\usepackage[margin=20mm]{geometry} 
\usepackage{amsmath,amsthm,amssymb}
\usepackage{graphicx}
\usepackage{listings}
\newenvironment{statement}[2][Statement]{\begin{trivlist}
\item[\hskip \labelsep {\bfseries #1}\hskip \labelsep {\bfseries #2.}]}{\end{trivlist}}
\usepackage{xcolor}

\usepackage{subfigure}


% Listings package for code rendering (No external dependencies)
\usepackage{listings}  
\usepackage{xcolor}   % Color support
\usepackage{tcolorbox} % Box for better appearance

% Define custom colors for code highlighting
\definecolor{codegreen}{rgb}{0,0.6,0}
\definecolor{codegray}{rgb}{0.5,0.5,0.5}
\definecolor{codepurple}{rgb}{0.58,0,0.82}
\definecolor{backcolour}{rgb}{0.95,0.95,0.92}


\lstset{frame=tb,
    language=Python,
    backgroundcolor=\color{backcolour},   
    commentstyle=\color{codegreen},
    keywordstyle=\color{magenta},
    numberstyle=\tiny\color{codegray},
    stringstyle=\color{codepurple},
    basicstyle=\ttfamily\footnotesize,
    breakatwhitespace=false,         
    breaklines=true,                 
    keepspaces=true,                 
    numbers=left,       
    numbersep=5pt,                  
    showspaces=false,                
    showstringspaces=false,
    showtabs=false,                  
    tabsize=2,
}




\title{Assignment 3}

\begin{document}
\maketitle


\section{1 Complex numbers}
%wanjing

\subsection{1.1}

\textbf{(i)} For 
\[
d = a + b, \quad \text{with } a = a_1 + i\,a_2,\; b = b_1 + i\,b_2,
\]
we have
\[
d = (a_1 + b_1) + i\,(a_2 + b_2).
\]

\textbf{(ii)} For 
\[
d = a - b,
\]
we obtain
\[
d = (a_1 - b_1) + i\,(a_2 - b_2).
\]

\textbf{(iii)} For the product 
\[
d = a \cdot b,
\]
we compute
\[
d = (a_1b_1 - a_2b_2) + i\,(a_1b_2 + a_2b_1).
\]

\textbf{(iv)} For the quotient 
\[
d = \frac{a}{b},
\]
first multiply the numerator and denominator by the complex conjugate of the denominator:
\[
\frac{a}{b} = \frac{a_1 + i\,a_2}{b_1 + i\,b_2}
= \frac{(a_1 + i\,a_2)(b_1 - i\,b_2)}{b_1^2 + b_2^2}.
\]
Expanding the numerator,
\[
(a_1 + i\,a_2)(b_1 - i\,b_2) = a_1b_1 + a_2b_2 + i\,(a_2b_1 - a_1b_2),
\]
so that
\[
d = \frac{a_1b_1 + a_2b_2}{b_1^2+b_2^2} + i\,\frac{a_2b_1 - a_1b_2}{b_1^2+b_2^2}.
\]

\subsection{1.2}

Since
\[
\sqrt{-3} = \sqrt{3}\,i,
\]
we can write
\[
d = 0 + i\,\sqrt{3}.
\]

\subsection{1.3}

Assume that the complex numbers are written as 
\[
a = a_r e^{i a_\theta} \quad \text{and} \quad b = b_r e^{i b_\theta}.
\]

\textbf{(i)} For the product 
\[
d = a \cdot b = a_r b_r e^{i(a_\theta + b_\theta)},
\]
the complex conjugate is
\[
\overline{a\cdot b} = a_r b_r e^{-i(a_\theta + b_\theta)}.
\]

\textbf{(ii)} For the quotient 
\[
d = \frac{a}{b} = \frac{a_r}{b_r} e^{i(a_\theta - b_\theta)},
\]
the complex conjugate is
\[
\overline{\frac{a}{b}} = \frac{a_r}{b_r} e^{-i(a_\theta - b_\theta)}.
\]

\subsection{1.4}

The complex conjugate is given by
\[
\overline{a_r e^{i a_\theta}} = a_r e^{-i a_\theta}.
\]

\subsection{1.5}

Let 
\[
a = a_r e^{i a_\theta} = a_r (\cos a_\theta + i\sin a_\theta)
\]
and
\[
b = b_r e^{i b_\theta} = b_r (\cos b_\theta + i\sin b_\theta).
\]

\textbf{(i)} For 
\[
d = a + b,
\]
we have
\[
d = \Bigl(a_r\cos a_\theta + b_r\cos b_\theta\Bigr) + i\Bigl(a_r\sin a_\theta + b_r\sin b_\theta\Bigr).
\]

\textbf{(ii)} For 
\[
d = a - b,
\]
we obtain
\[
d = \Bigl(a_r\cos a_\theta - b_r\cos b_\theta\Bigr) + i\Bigl(a_r\sin a_\theta - b_r\sin b_\theta\Bigr).
\]

\subsection{1.6}

The polar form is given by
\[
a = r\, e^{i\theta},
\]
where
\[
r = \sqrt{a_1^2 + a_2^2} \quad \text{and} \quad \theta = \arctan\!\left(\frac{a_2}{a_1}\right)
\]
(with appropriate adjustments depending on the quadrant). Thus,
\[
a = \sqrt{a_1^2+a_2^2}\; e^{i\arctan\!\left(\frac{a_2}{a_1}\right)}.
\]


\section{2 Fourier Transform – Theory}
%zhigao

\section{3 Fourier Transform – In Practice}
%shuangcheng



\end{document}
