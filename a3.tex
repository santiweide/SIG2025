\documentclass[12pt]{article}
\usepackage[utf8]{inputenc}
\usepackage{upquote}
\usepackage[margin=20mm]{geometry} 
\usepackage{amsmath,amsthm,amssymb}
\usepackage{graphicx}
\usepackage{listings}
\newenvironment{statement}[2][Statement]{\begin{trivlist}
\item[\hskip \labelsep {\bfseries #1}\hskip \labelsep {\bfseries #2.}]}{\end{trivlist}}
\usepackage{xcolor}

\usepackage{subfigure}


% Listings package for code rendering (No external dependencies)
\usepackage{listings}  
\usepackage{xcolor}   % Color support
\usepackage{tcolorbox} % Box for better appearance
\usepackage{enumitem} 
% Define custom colors for code highlighting
\definecolor{codegreen}{rgb}{0,0.6,0}
\definecolor{codegray}{rgb}{0.5,0.5,0.5}
\definecolor{codepurple}{rgb}{0.58,0,0.82}
\definecolor{backcolour}{rgb}{0.95,0.95,0.92}


\lstset{frame=tb,
    language=Python,
    backgroundcolor=\color{backcolour},   
    commentstyle=\color{codegreen},
    keywordstyle=\color{magenta},
    numberstyle=\tiny\color{codegray},
    stringstyle=\color{codepurple},
    basicstyle=\ttfamily\footnotesize,
    breakatwhitespace=false,         
    breaklines=true,                 
    keepspaces=true,                 
    numbers=left,       
    numbersep=5pt,                  
    showspaces=false,                
    showstringspaces=false,
    showtabs=false,                  
    tabsize=2,
}





\title{Assignment 3}


\begin{document}
\maketitle

\section{1 Complex numbers}
%wanjing

\subsection{} % 1.1

\textbf{(i)} 

\begin{equation}
\begin{aligned}
d &=(a_1 + i a_2) + (b_1 + i b_2) = (a_1 + b_1) + i\,(a_2 + b_2).
\end{aligned}
\end{equation}

\textbf{(ii)}
\begin{equation}
\begin{aligned}
d &=(a_1 + i a_2) - (b_1 + i b_2) = (a_1 - b_1) + i\,(a_2 - b_2).
\end{aligned}
\end{equation}

\textbf{(iii)} 
\begin{equation}
\begin{aligned}
d &=(a_1 + i a_2)(b_1 + i b_2) \\
&= a_1b_1 + i a_1b_2 + i a_2b_1 + i^2 a_2b_2\\
&= (a_1b_1 - a_2b_2) + i\,(a_1b_2 + a_2b_1).
\end{aligned}
\end{equation}

\textbf{(iv)} For the quotient 
\begin{equation}
\begin{aligned}
d &= \frac{a}{b} \\
&= \frac{a_1 + i\,a_2}{b_1 + i\,b_2}\\
&= \frac{(a_1 + i\,a_2)(b_1 - i\,b_2)}{(b_1 + i\,b_2)(b_1 - i\,b_2)}\\
&= \frac{(a_1 + i\,a_2)(b_1 - i\,b_2)}{b_1^2 + b_2^2} \\
&= \frac{a_1b_1 + a_2b_2 + i\,(a_2b_1 - a_1b_2)}{b_1^2 + b_2^2}\\
&= \frac{a_1b_1 + a_2b_2}{b_1^2+b_2^2} + i\,\frac{a_2b_1 - a_1b_2}{b_1^2+b_2^2}.
\end{aligned}
\end{equation}


\subsection{} % 1.2
by $\sqrt{-3} = \sqrt{3}\,i$ , we have $d = 0 + i\,\sqrt{3}$. 

\subsection{} % 1.3
\textbf{(i)} 
\begin{equation}
\begin{aligned}
d &= a \cdot b = a_r e^{i a_\theta} \cdot b_r e^{i b_\theta} = a_r b_r e^{i(a_\theta + b_\theta)}
\end{aligned}
\end{equation}


\textbf{(ii)} 

\begin{equation}
\begin{aligned}
d &=  \frac{a}{b} = \frac{a_r e^{i a_\theta}}{ b_r e^{i b_\theta}} =  \frac{a_r}{b_r} e^{i(a_\theta - b_\theta)}
\end{aligned}
\end{equation}

\subsection{} %1.4
\begin{equation}
\begin{aligned}
\overline{a_r e^{i a_\theta}} = a_r e^{-i a_\theta}.
\end{aligned}
\end{equation}

\subsection{}

Let 
\[
a = a_r e^{i a_\theta} = a_r (\cos a_\theta + i\sin a_\theta)
\]
and
\[
b = b_r e^{i b_\theta} = b_r (\cos b_\theta + i\sin b_\theta).
\]

\textbf{(i)}
\begin{equation}
\begin{aligned}
d &= a + b\\
& = \Bigl(a_r\cos a_\theta + b_r\cos b_\theta\Bigr) + i\Bigl(a_r\sin a_\theta + b_r\sin b_\theta\Bigr).
\end{aligned}
\end{equation}


\textbf{(ii)} For 
\begin{equation}
\begin{aligned}
d &= a + b\\
& = \Bigl(a_r\cos a_\theta - b_r\cos b_\theta\Bigr) + i\Bigl(a_r\sin a_\theta - b_r\sin b_\theta\Bigr).
\end{aligned}
\end{equation}
\subsection{}

The polar form is given by
\[
a = r\, e^{i\theta},
\]
where
\[
r = \sqrt{a_1^2 + a_2^2} \quad \text{and} \quad \theta = \arctan\!\left(\frac{a_2}{a_1}\right)
\]
So
\[
a = \sqrt{a_1^2+a_2^2}\; e^{i\arctan\!\left(\frac{a_2}{a_1}\right)}.
\]


\section{2 Fourier Transform – Theory}
%zhigao
\subsection{}
The Fourier Transform is:
\[F\{f\}(u) = F(u) = \int_{-\infty}^{\infty} f(x) \exp(-i 2\pi ux) \,dx\]
By using the Euler's formula(\(\exp(-i 2\pi ux) = \cos(2\pi ux) - i\sin(2\pi ux).\)), This formula can be rewritten as:
\[
F(u) = \int_{-\infty}^{\infty} f(x) \cos(2\pi ux) \,dx - i \int_{-\infty}^{\infty} f(x) \sin(2\pi ux) \,dx.
\]
\(\sin(2\pi ux)\) is an odd function that integrates to 0 on the symmetric interval.
\[
\int_{-\infty}^{\infty} f(x) \sin(2\pi ux) \,dx = 0.
\]
Hence, The imaginary part of \(F(u)\) is 0. \(F(u)\) is real.
\newline
Then, To verify that \(F(u)\) is an even function. I need to prove \(F(u) = F(-u)\).
Firstly, I got:
\[
F(-u) = \int_{-\infty}^{\infty} f(x) \exp(i 2\pi ux) \,dx.
\]
Then I can replace the variable \(-x\) to \(x'\), hence \(dx' = -dx\), so I got:
\[
F(-u) = \int_{\infty}^{-\infty} f(-x') \exp(-i 2\pi ux') (-dx') 
\]
From the question, the \(f\) is an even function, so \(f(-x')=f(x')\).
\(F(-u)\) can be rewritten as:

\[F(-u) = \int_{-\infty}^{\infty} f(x') \exp(-i 2\pi ux') \,dx' = F(u).\]
In summary, an even, real function is also even, real after Fourier transformed.
\subsection{}
\begin{enumerate}[label=(\roman*)]
  \item \[
\begin{aligned}
\int_{-\infty}^{\infty} b_a(x)\,dx 
&= \int_{-a/2}^{a/2} \frac{1}{a}\,dx  \\
&= \frac{1}{a} \times \Bigl(\tfrac{a}{2} - \bigl(-\tfrac{a}{2}\bigr)\Bigr) \\
&= \frac{1}{a} \times a \\
&= 1.
\end{aligned}
\]
Since the box function has values of \(\frac{1}{a}\) only on \([-\frac{a}{2},\frac{a}{2}]\), the upper and lower limits of the integral can be changed to \([-\frac{a}{2},\frac{a}{2}]\).
  \item
  By using the definition of the Fourier transform given in Bracewell Chapter 2:
  \[
    \int_{-\infty}^{\infty} f(x)\,\exp(-i2 \pi xs)\,dx.
    \]
  the box function could be written as:
    \[
B_a(k) 
= \int_{-\infty}^{\infty} b_a(x)\,\exp(-i2\pi kx)\,dx 
= \int_{-a/2}^{\,a/2} \frac{1}{a}\,\exp(-i2\pi kx)\,dx.
\]
Continue calculating:
\[
\begin{aligned}
B_a(k)
&= \frac{1}{a} \int_{-a/2}^{\,a/2} \exp(-i2\pi k x)\,dx \\[6pt]
&= \frac{1}{a} \,\biggl[\frac{\exp(-i2\pi kx)}{-i2\pi k}\biggr]_{x=-a/2}^{\,x=a/2}
&= \frac{1}{a} \cdot \frac{1}{-\,i\,2\pi\,k}
\bigl(\,\exp\bigl[-\,i\,2\pi\,k \cdot \tfrac{a}{2}\bigr]
 \;-\; \exp\bigl[-\,i\,2\pi\,k \cdot \bigl(-\tfrac{a}{2}\bigr)\bigr]\bigr).
\end{aligned}
\]
By the Euler's formula:
\[
e^{i\theta} = \cos(\theta) + i \sin(\theta), e^{-i\theta} = \cos(\theta) - i \sin(\theta)
\]
The function above could be rewritten as:
\[
B_a(k) 
= \frac{1}{a} \,\frac{-\,2\,i\,\sin\bigl(\pi\,a\,k\bigr)}{-\,i\,2\pi\,k}. = \frac{1}{\pi\,k\,a}\,\sin\bigl(\pi\,a\,k\bigr)
\]
Using the \(sinc(x) = \frac{sinx}{x}\), the final form of the Fourier transform of the box function is:
\[B_a(k) = sinc(\pi k a)\]

  \item 
  If \(a \rightarrow 0\), \(sin(ak\pi)\rightarrow 0\), and we know 
  \[
\lim_{z \to 0} \frac{\sin z}{z} = 1.
\]
 we can see \(ak\pi\) to \(z\), hence we get:
  \[
\lim_{a \to 0} \frac{\sin (ak\pi)}{ak\pi} = 1.
\]
  \item 
  I think if the box function is narrow in x space(a is near 0),  its Fourier transform will be wide in the k space(frequency domain). Conversely, if a is larger, then it is wider in x-space and narrower in k-space. This is the uncertainty property of the Fourier transform. this means a narrow function has a wide Fourier transform, and a wide function has a narrow Fourier transform.
\end{enumerate}
\section{3 Fourier Transform – In Practice}
%shuangcheng


\end{document}
